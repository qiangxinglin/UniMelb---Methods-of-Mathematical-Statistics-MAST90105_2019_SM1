\documentclass[]{article}
\usepackage{lmodern}
\usepackage{amssymb,amsmath}
\usepackage{ifxetex,ifluatex}
\usepackage{fixltx2e} % provides \textsubscript
\ifnum 0\ifxetex 1\fi\ifluatex 1\fi=0 % if pdftex
  \usepackage[T1]{fontenc}
  \usepackage[utf8]{inputenc}
\else % if luatex or xelatex
  \ifxetex
    \usepackage{mathspec}
  \else
    \usepackage{fontspec}
  \fi
  \defaultfontfeatures{Ligatures=TeX,Scale=MatchLowercase}
\fi
% use upquote if available, for straight quotes in verbatim environments
\IfFileExists{upquote.sty}{\usepackage{upquote}}{}
% use microtype if available
\IfFileExists{microtype.sty}{%
\usepackage{microtype}
\UseMicrotypeSet[protrusion]{basicmath} % disable protrusion for tt fonts
}{}
\usepackage[margin=1in]{geometry}
\usepackage{hyperref}
\hypersetup{unicode=true,
            pdftitle={Assignment 1},
            pdfauthor={Vinh Nguyen (ID 1029531)},
            pdfborder={0 0 0},
            breaklinks=true}
\urlstyle{same}  % don't use monospace font for urls
\usepackage{color}
\usepackage{fancyvrb}
\newcommand{\VerbBar}{|}
\newcommand{\VERB}{\Verb[commandchars=\\\{\}]}
\DefineVerbatimEnvironment{Highlighting}{Verbatim}{commandchars=\\\{\}}
% Add ',fontsize=\small' for more characters per line
\usepackage{framed}
\definecolor{shadecolor}{RGB}{248,248,248}
\newenvironment{Shaded}{\begin{snugshade}}{\end{snugshade}}
\newcommand{\AlertTok}[1]{\textcolor[rgb]{0.94,0.16,0.16}{#1}}
\newcommand{\AnnotationTok}[1]{\textcolor[rgb]{0.56,0.35,0.01}{\textbf{\textit{#1}}}}
\newcommand{\AttributeTok}[1]{\textcolor[rgb]{0.77,0.63,0.00}{#1}}
\newcommand{\BaseNTok}[1]{\textcolor[rgb]{0.00,0.00,0.81}{#1}}
\newcommand{\BuiltInTok}[1]{#1}
\newcommand{\CharTok}[1]{\textcolor[rgb]{0.31,0.60,0.02}{#1}}
\newcommand{\CommentTok}[1]{\textcolor[rgb]{0.56,0.35,0.01}{\textit{#1}}}
\newcommand{\CommentVarTok}[1]{\textcolor[rgb]{0.56,0.35,0.01}{\textbf{\textit{#1}}}}
\newcommand{\ConstantTok}[1]{\textcolor[rgb]{0.00,0.00,0.00}{#1}}
\newcommand{\ControlFlowTok}[1]{\textcolor[rgb]{0.13,0.29,0.53}{\textbf{#1}}}
\newcommand{\DataTypeTok}[1]{\textcolor[rgb]{0.13,0.29,0.53}{#1}}
\newcommand{\DecValTok}[1]{\textcolor[rgb]{0.00,0.00,0.81}{#1}}
\newcommand{\DocumentationTok}[1]{\textcolor[rgb]{0.56,0.35,0.01}{\textbf{\textit{#1}}}}
\newcommand{\ErrorTok}[1]{\textcolor[rgb]{0.64,0.00,0.00}{\textbf{#1}}}
\newcommand{\ExtensionTok}[1]{#1}
\newcommand{\FloatTok}[1]{\textcolor[rgb]{0.00,0.00,0.81}{#1}}
\newcommand{\FunctionTok}[1]{\textcolor[rgb]{0.00,0.00,0.00}{#1}}
\newcommand{\ImportTok}[1]{#1}
\newcommand{\InformationTok}[1]{\textcolor[rgb]{0.56,0.35,0.01}{\textbf{\textit{#1}}}}
\newcommand{\KeywordTok}[1]{\textcolor[rgb]{0.13,0.29,0.53}{\textbf{#1}}}
\newcommand{\NormalTok}[1]{#1}
\newcommand{\OperatorTok}[1]{\textcolor[rgb]{0.81,0.36,0.00}{\textbf{#1}}}
\newcommand{\OtherTok}[1]{\textcolor[rgb]{0.56,0.35,0.01}{#1}}
\newcommand{\PreprocessorTok}[1]{\textcolor[rgb]{0.56,0.35,0.01}{\textit{#1}}}
\newcommand{\RegionMarkerTok}[1]{#1}
\newcommand{\SpecialCharTok}[1]{\textcolor[rgb]{0.00,0.00,0.00}{#1}}
\newcommand{\SpecialStringTok}[1]{\textcolor[rgb]{0.31,0.60,0.02}{#1}}
\newcommand{\StringTok}[1]{\textcolor[rgb]{0.31,0.60,0.02}{#1}}
\newcommand{\VariableTok}[1]{\textcolor[rgb]{0.00,0.00,0.00}{#1}}
\newcommand{\VerbatimStringTok}[1]{\textcolor[rgb]{0.31,0.60,0.02}{#1}}
\newcommand{\WarningTok}[1]{\textcolor[rgb]{0.56,0.35,0.01}{\textbf{\textit{#1}}}}
\usepackage{graphicx,grffile}
\makeatletter
\def\maxwidth{\ifdim\Gin@nat@width>\linewidth\linewidth\else\Gin@nat@width\fi}
\def\maxheight{\ifdim\Gin@nat@height>\textheight\textheight\else\Gin@nat@height\fi}
\makeatother
% Scale images if necessary, so that they will not overflow the page
% margins by default, and it is still possible to overwrite the defaults
% using explicit options in \includegraphics[width, height, ...]{}
\setkeys{Gin}{width=\maxwidth,height=\maxheight,keepaspectratio}
\IfFileExists{parskip.sty}{%
\usepackage{parskip}
}{% else
\setlength{\parindent}{0pt}
\setlength{\parskip}{6pt plus 2pt minus 1pt}
}
\setlength{\emergencystretch}{3em}  % prevent overfull lines
\providecommand{\tightlist}{%
  \setlength{\itemsep}{0pt}\setlength{\parskip}{0pt}}
\setcounter{secnumdepth}{0}
% Redefines (sub)paragraphs to behave more like sections
\ifx\paragraph\undefined\else
\let\oldparagraph\paragraph
\renewcommand{\paragraph}[1]{\oldparagraph{#1}\mbox{}}
\fi
\ifx\subparagraph\undefined\else
\let\oldsubparagraph\subparagraph
\renewcommand{\subparagraph}[1]{\oldsubparagraph{#1}\mbox{}}
\fi

%%% Use protect on footnotes to avoid problems with footnotes in titles
\let\rmarkdownfootnote\footnote%
\def\footnote{\protect\rmarkdownfootnote}

%%% Change title format to be more compact
\usepackage{titling}

% Create subtitle command for use in maketitle
\newcommand{\subtitle}[1]{
  \posttitle{
    \begin{center}\large#1\end{center}
    }
}

\setlength{\droptitle}{-2em}

  \title{Assignment 1}
    \pretitle{\vspace{\droptitle}\centering\huge}
  \posttitle{\par}
    \author{Vinh Nguyen (ID 1029531)}
    \preauthor{\centering\large\emph}
  \postauthor{\par}
    \date{}
    \predate{}\postdate{}
  

\begin{document}
\maketitle

\begin{enumerate}
\def\labelenumi{\arabic{enumi}.}
\tightlist
\item
  Five fair coins are tossed (i.e., the probability of tail and head
  equals 0.5 for each coin) and the number of tails, T is counted. Find
  the conditional probability that T ≥ 1 given that at least one coin
  shows head.\\
  H: number of heads\\
  H + T = 5 (5 tosses)\\
  P(at least one coin shows head) = P(H \textgreater{}= 1) = P(T
  \textless{}= 4)\\
  \textbf{P(T ≥ 1\textbar{}T \textless{}= 4)} = P(T ≥ 1 and T
  \textless{}= 4) / P(T \textless{}= 4) = P(1 \textless{}= T
  \textless{}= 4) / P(T \textless{}= 4)\\
  T \textasciitilde{} Bi(n=5, p=0.5)
\end{enumerate}

\begin{Shaded}
\begin{Highlighting}[]
\NormalTok{p1 <-}\StringTok{ }\KeywordTok{sum}\NormalTok{(}\KeywordTok{dbinom}\NormalTok{(}\DecValTok{1}\OperatorTok{:}\DecValTok{4}\NormalTok{, }\DecValTok{5}\NormalTok{, }\FloatTok{0.5}\NormalTok{)) }\CommentTok{# P(1 <= T <= 4)}
\NormalTok{p2 <-}\StringTok{ }\KeywordTok{pbinom}\NormalTok{(}\DecValTok{4}\NormalTok{, }\DataTypeTok{size=}\DecValTok{5}\NormalTok{, }\DataTypeTok{prob=}\FloatTok{0.5}\NormalTok{) }\CommentTok{# P(T <= 4)}
\StringTok{"P(T ≥ 1|T <= 4)"}
\end{Highlighting}
\end{Shaded}

\begin{verbatim}
## [1] "P(T ≥ 1|T <= 4)"
\end{verbatim}

\begin{Shaded}
\begin{Highlighting}[]
\NormalTok{p1}\OperatorTok{/}\NormalTok{p2}
\end{Highlighting}
\end{Shaded}

\begin{verbatim}
## [1] 0.9677419
\end{verbatim}

\begin{enumerate}
\def\labelenumi{\arabic{enumi}.}
\setcounter{enumi}{1}
\tightlist
\item
  Birthday paradox\\
\end{enumerate}

\begin{enumerate}
\def\labelenumi{(\alph{enumi})}
\tightlist
\item
  Consider a group of 3 students. Each student has a birthday that can
  be any one of the days numbered 1, 2, 3, \ldots{}, 365. What is the
  probability that none of them have the same birthday with each
  other?\\
  Student 1: can take 365 possible birthdays\\
  Student 2: can take 364 possible birthdays\\
  Student 3: can take 363 possible birthdays\\
  All possibilites: 365 x 365 x 365\\
  P(no same birthday) = (365 x 364 x 363) / (365 x 365 x 365)
\end{enumerate}

\begin{Shaded}
\begin{Highlighting}[]
\StringTok{"P(no same birthday)"}
\end{Highlighting}
\end{Shaded}

\begin{verbatim}
## [1] "P(no same birthday)"
\end{verbatim}

\begin{Shaded}
\begin{Highlighting}[]
\NormalTok{(}\DecValTok{365}\OperatorTok{*}\DecValTok{364}\OperatorTok{*}\DecValTok{363}\NormalTok{) }\OperatorTok{/}\StringTok{ }\DecValTok{365}\OperatorTok{^}\DecValTok{3}
\end{Highlighting}
\end{Shaded}

\begin{verbatim}
## [1] 0.9917958
\end{verbatim}

\begin{enumerate}
\def\labelenumi{(\alph{enumi})}
\setcounter{enumi}{1}
\tightlist
\item
  Consider a group of 23 students. Each student has a birthday that can
  be any one of the days numbered 1, 2, 3, \ldots{}, 365.\\
\end{enumerate}

\begin{enumerate}
\def\labelenumi{\roman{enumi}.}
\tightlist
\item
  What is the probability that none of them have the same birthday with
  each other?\\
  Student 1: can take 365 possible birthdays\\
  Student 2: can take (365-1) possible birthdays\\
  \ldots{}\\
  Student 23: can take (365-22) possible birthdays\\
  All possibilites: 365\^{}23
\end{enumerate}

\begin{Shaded}
\begin{Highlighting}[]
\StringTok{"P(no same birthday)"}
\end{Highlighting}
\end{Shaded}

\begin{verbatim}
## [1] "P(no same birthday)"
\end{verbatim}

\begin{Shaded}
\begin{Highlighting}[]
\NormalTok{p_null <-}\StringTok{ }\KeywordTok{prod}\NormalTok{((}\DecValTok{365-22}\NormalTok{)}\OperatorTok{:}\DecValTok{365}\NormalTok{) }\OperatorTok{/}\StringTok{ }\DecValTok{365}\OperatorTok{^}\DecValTok{23}
\NormalTok{p_null}
\end{Highlighting}
\end{Shaded}

\begin{verbatim}
## [1] 0.4927028
\end{verbatim}

\begin{enumerate}
\def\labelenumi{\roman{enumi}.}
\setcounter{enumi}{1}
\tightlist
\item
  What is the probability that some of them have the same birthday with
  each other? Is this probability greater than 0.5?\\
  P(some same birthdays) = 1 - P(no same birthday)
\end{enumerate}

\begin{Shaded}
\begin{Highlighting}[]
\StringTok{"P(some same birthdays)"}
\end{Highlighting}
\end{Shaded}

\begin{verbatim}
## [1] "P(some same birthdays)"
\end{verbatim}

\begin{Shaded}
\begin{Highlighting}[]
\NormalTok{p_some <-}\StringTok{ }\DecValTok{1} \OperatorTok{-}\StringTok{ }\NormalTok{p_null}
\NormalTok{p_some}
\end{Highlighting}
\end{Shaded}

\begin{verbatim}
## [1] 0.5072972
\end{verbatim}

\begin{Shaded}
\begin{Highlighting}[]
\StringTok{"Is this probability greater than 0.5?"}
\end{Highlighting}
\end{Shaded}

\begin{verbatim}
## [1] "Is this probability greater than 0.5?"
\end{verbatim}

\begin{Shaded}
\begin{Highlighting}[]
\NormalTok{p_some }\OperatorTok{>}\StringTok{ }\FloatTok{0.5}
\end{Highlighting}
\end{Shaded}

\begin{verbatim}
## [1] TRUE
\end{verbatim}


\end{document}
